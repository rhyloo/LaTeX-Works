\documentclass[a4paper,12pt]{article}
\usepackage[utf8]{inputenc}
\usepackage[T1]{fontenc}
\usepackage[spanish]{babel}
\usepackage{amsmath}
\usepackage{amsfonts}
\usepackage{fancyhdr}
\usepackage{amssymb}
\usepackage[font=scriptsize,labelfont=bf]{caption}
\usepackage{lmodern}
\usepackage{xcolor} 
\usepackage{mdframed}
\usepackage{wrapfig}
\usepackage{multirow} 
\usepackage{lipsum}
\usepackage{multicol} 
\usepackage{scrextend} 
\usepackage{tikz}
\usetikzlibrary{positioning}
\usepackage{float}
\usepackage{graphicx}
\usepackage{tikzpagenodes}
\usepackage[absolute]{textpos} 
\usepackage{colortbl}
\usepackage{array}
\usepackage[hidelinks]{hyperref}
\usepackage[makeroom]{cancel}
\usepackage{eso-pic}
\usepackage[a4paper,left=3cm,right=2cm,top=2.5cm,bottom=2.5cm]{geometry}
\usepackage[final]{pdfpages}
\usepackage{float}
\usepackage{fancyhdr}
\pagestyle{fancy}
\fancyhf{}
\lhead{Sistemas electrónicos}
\rhead{Jorge Benavides Macías}
\rfoot{\thepage}
\renewcommand{\headrulewidth}{0.4pt}% Default \headrulewidth is 0.4pt
\setlength{\headheight}{16pt}
\setlength{\parskip}{1em}
\setlength{\parindent}{0em}
\definecolor{my_blue}{RGB}{0, 46, 93}
\definecolor{my_white}{RGB}{255, 255, 255}
\definecolor{Prune}{RGB}{0,46,93}

\newcommand*{\dummypic}[4]{%
\setlength{\fboxsep}{0pt}%
\begin{minipage}[t][0.2\textwidth][c]{0.25\textwidth}
\centering \includegraphics[#1]{#2}
\caption{#3}
\label{#4}
\end{minipage}%
}

\Style{2} %language 1 english 2 spanish
\begin{document}
%------------------------------------------
% information
\title{\textbf{Carne de Lasagna}\footnotesize{Rhyloo}}
\PrepTime{45}%preparation time
\CookingTime{30} %cooking time
\CookingTempe{0} %optional (with 0) cooking temperature
\TypeCooking{} %type of cookings
\NPerson{4} %number of persons
%------------------------------------------

\begin{ingredient}
\begin{main}
\item Carne molida (cerdo y vacuno mezclada al 50\%) - 500 gr.
\item Pimiento verde - 2
\item Pimiento rojo - 2
\item Cebolla - 2
\item Tomate frito - 1 bote
\item Sal fina
\item Pimienta negra (molinillo)
\end{main}
\begin{subingredient}{Opcional}
\item Queso mozarella
\item Orégano
\item Calabacín
\item Benrenjena
\item Zanahoria
\item Pasta, arroz, patatas, o cualquier carbohidrato.
\end{subingredient}
\end{ingredient} %no space with \begin{recipe}
\begin{recipe}
\step{Poner una olla mediana, a fuego medio, con un chorro de aceite. Esperar a que el aceite esté caliente.}
\step{Picar la cebolla en cuadraditos.}
\step{Poner la olla a fuego bajo, echar la cebolla en la olla, remover cada cierto tiempo. Estará lista cuando alcance un color amarillento, esté blandita y desprenda un olor dulce.}
\step{Picar los pimienos en cuadraditos, lo más pequeño posible. Echar en la olla y salpimentar}
\textbf{Para asegurar su cocción se puede tapar, el pimiento desprende mucha agua por lo tanto hay que vigilarlo ya que el agua eliminaría la sal y la pimienta, si se nos pasa, destapar, esperar a que se evapore un poco, salpimentar.\\}
\step{Preparar la carne para su cocción, la aplastamos en una tabla de picar, salpimentamos y empezamos a mezclarla cual plastilina, la aplastamos otra vez, salpimentamos, mezclamos y extraemos trozos.}
\step{Los trozos del paso anterior los echamos en la olla, deberían empezar a cocinarse, la carne cambia de color fácilmente pasa de rojo a gris, antes de que esté completamente hecha trozeamos con una espátula hasta conseguir trozos muy pequeños.}
\step{En este punto podemos hacer uso del orégano para que adquiera un sabor especial.}
\step{Echamos el tomate frito, lentamente mientras removemos, debemos cubrir totalmente la carne y dejar un poco más para que se cocine a fuego lento.}
\step{Añadimos más oregano. Tapamos y esperar entre 10 y 15 minutos.}
\end{recipe}

\begin{notes}
Es conveniente sacar la carne con tiempo; 24 horas antes si es invierno y 12 horas si es verano, siempre en la nevera para evitar malos olores en casa o que se ponga mala, mantener la cadena del frío es importante.\\
\textbf{Los pimientos tiene mucha agua, cuidado al cocinarlos.}\\
Para comerla se puede combinar con distintos carbohidratos, \textbf{recomiendo la pasta}, las verduras no pueden faltar, calabacín y berenjena cocida como elementos perfectos.\\
El queso no hace falta derretirlo en el microondas, si se ha hecho bien, la carne estará caliente y por lo tanto el queso se derretirá.
\end{notes}
\end{document}
