%%%%%%%%%%%%%%%%%%%%%%%% 
% GLOSARIOS
%%%%%%%%%%%%%%%%%%%%%%%%
\usepackage[acronym,nonumberlist,toc]{glossaries}
\usepackage{glossary-superragged}
\newglossarystyle{modsuper}{%
  \setglossarystyle{super}%
  \renewcommand{\glsgroupskip}{}
}
\renewcommand{\glsnamefont}[1]{\textbf{#1}}

\usepackage{lipsum}
%%%%%%%%%%%%%%%%%%%%%%%%
% DOCUMENTO EN ESPAÑOL
%%%%%%%%%%%%%%%%%%%%%%%%
\usepackage[base]{babel}
\usepackage{polyglossia}
\setdefaultlanguage{spanish}

\addto\captionsspanish{%
	\renewcommand{\listtablename}{Índice de tablas}
	\renewcommand{\tablename}{Tabla}
	\renewcommand{\lstlistingname}{Código}
	\renewcommand{\lstlistlistingname}{Índice de \lstlistingname s}
	\renewcommand{\glossaryname}{Glosario}
	\renewcommand{\acronymname}{Acrónimos}
	\renewcommand{\bibname}{Bibliografía}%
}
% Paquetes
\usepackage[
  inner	=	3.0cm, % Margen interior
  outer	=	2.5cm, % Margen exterior
  top	=	2.5cm, % Margen superior
  bottom=	2.5cm, % Margen inferior
  includeheadfoot, % Incluye cabecera y pie de página en los márgenes
]{geometry}
\newgeometry{ignoreall,top=2cm,outer=2cm,inner=2cm}
\thispagestyle{empty}
\usepackage{graphicx}
\usepackage{calc}
\usepackage{xstring}
\usepackage{setspace}
\usepackage{ifthen}
\usepackage{tikz}
\usetikzlibrary{calc}
\usetikzlibrary{positioning}
\usetikzlibrary{fit}
\usetikzlibrary{shapes}
\usetikzlibrary{arrows}
\usepackage{eso-pic}
%%%%%%%%%%%%%%%%%%%%%%%%
% TEXTO
%%%%%%%%%%%%%%%%%%%%%%%%
% Paquete para poder modificar las fuente de texto
\usepackage{xltxtra}
% Cualquier tamaño de texto. Uso: {\fontsize{100pt}{120pt}\selectfont tutexto}
\usepackage{anyfontsize}
% Para modificar parametros del texto.
\usepackage{setspace}
% Paquete para posicionar bloques de texto
\usepackage{textpos}
% Paquete para realizar cajas de texto.
% Uso: \begin{mdframed}[linecolor=red!100!black] tutexto \end{mdframed}
\usepackage{framed,mdframed}
% Para subrayar. Uso: \hlc[tucolor]{tutexto}
\newcommand{\hlc}[2][yellow]{ {\sethlcolor{#1} \hl{#2}} }
\newfontfamily\FuenteTitulo{HelveticaLTStd-Cond}[Path=./include/fuentes/]
% Helvetica. Uso: {\FuentePortada tutexto}
\newfontfamily\FuentePortada{Helvetica}[Path=./include/fuentes/]

\makeatletter
\def\identificador#1{\def\@identificador{#1}}
\def\identificadorx{\@identificador}% Display title

\def\seguridad#1{\def\@seguridad{#1}}
\def\seguridadx{\@seguridad}% Display title

\def\titulo#1{\def\@titulo{#1}}
\def\titulox{\@titulo}% Display title

\def\tipo#1{\def\@tipo{#1}}
\def\tipox{\@tipo}% Display title

\newcommand*{\logo}[2][scale=0.5]{%
  \gdef\@logo@params{#1}%
  \gdef\@logo{#2}%
}
\newcommand*{\@logo@params}{}
\newcommand*{\@logo}{}
\def\logox{\expandafter\includegraphics\expandafter[\@logo@params]{\@logo}}

\def\autor#1{\def\@autor{#1}}
\def\autorx{\@autor}% Display title

\def\tutor#1{\def\@tutor{#1}}
\def\tutorx{\@tutor}% Display title

\def\fecha#1{\def\@fecha{#1}}
\def\fechax{\@fecha}% Display title

 \makeatother


\StrLen{\@titulox}[\longitudtitulo] % Cuenta los caracteres título
\definecolor{fondo}{RGB}{32,2,116}
\newcommand{\colorfondo}{white}



% Tamaño por defecto de la fuente de texto para:
\def\FuenteTamano{55pt}	% Tamaño para el título del trabajo
\def\interlinportada{5.0} % Interlineado por defecto para el título
\def\TamTrabajo{20pt} 	% Tamaño para el tipo de trabajo (grado o máster)
\def\TamTrabajoIn{20pt} 	% Tamaño para el salto de línea después de tipo de trabajo
\def\TamOtros{14pt} 	% Tamaño para datos personales y fecha
\def\TamOtrosIn{1pt} 	% Tamaño para los saltos de línea en la info personal


% Comprueba la longitud del título y según sea este determina unos valores nuevos
\ifthenelse{\longitudtitulo > 180}{
\def\FuenteTamano{35pt}		% Si es mayor a 180 caracteres tamaño de fuente 35pt
\def\interlinportada{3.5}} 	% Establece nuevo interlineado
{\ifthenelse{\longitudtitulo > 140}{
\def\FuenteTamano{40pt}		% Si es mayor a 140 caracteres tamaño de fuente 40pt
\def\interlinportada{4.0}} 	% Establece nuevo interlineado
{\ifthenelse{\longitudtitulo > 120}{
\def\FuenteTamano{50pt}		% Si es mayor a 120 caracteres tamaño de fuente 50pt
\def\interlinportada{4.5}} 	% Establece nuevo interlineado
{} % Si no, no modifica el tamaño
} }


