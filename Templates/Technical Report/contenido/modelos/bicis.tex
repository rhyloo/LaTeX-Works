\subsubsection{Bicicletas eléctricas}
%\paragraph{Descripción general}
% \paragraph{Estudio del reparto.}
%  %Escoger modelo
% \paragraph{Mantenimiento y robustez.}
% \paragraph{Viabilidad económica}
% \paragraph{Conclusiones modelo.}

\paragraph{Consideraciones previas}

Esto debe ir como ANEXO, las bicis son considerados vehículos HÍBRIDOS, no eléctricos

Las bicicletas de pedaleo deben de llevar siempre pedales y no pueden funcionar con un acelerador, por lo que es totalmente ilegal conducir por la vía pública las ``mini-bicicleta" sin pedales, según recoge el BOE y la normativa de circulación.  Según la normativa vigente, excluiremos del estudio este tipo de vehículo.

Las bicicletas eléctricas propiamente hablando son reconocidas como ``ciclo de pedaleo asistido" según el BOE N.º 124\cite{boe124}, lo que implica que aunque tienen motor eléctrico auxiliar, no pueden ser propulsadas exclusivamente haciendo uso del mismo, por lo que son vehículos híbridos entre eléctrico y manual.

\paragraph{Modelo F.Lli Schiano E-Moon}

Actualmente, una de las mejores bicicletas eléctricas del mercado calidad-precio que cumple con todas las normativas vigentes es el modelo F.Lli Schiano E-Moon, que está a la venta por 729 \glssymbol{euro}.

Respecto a sus características, posee un motor delantero ANANDA M129F de 250 \glssymbol{vatios} de potencia y 36 \glssymbol{voltios}, que permite alcanzar velocidades de hasta 25 \glssymbol{velocidad}. Presenta una batería de litio GREENWAY YJ145 36\glssymbol{voltios} 13\glssymbol{amperiohora} 468\glssymbol{vatiohora}, con una vida útil de entre 4 y 6 años y una autonomía de hasta 120 \glssymbol{km} con 6 horas de carga completa. 

Su sistema de frenado es V-Brake en las ruedas delantera y trasera, ambos neumáticos son Kenda de 26 pulgadas. El cuerpo de la bicicleta no es plegable, es de una aleación de aluminio resistente, aunque no ligera (25 \glssymbol{km}). Consta de una caja de cambios Shimano Tourney TY21 de 7 velocidades, un timbre, una luz delantera y otra trasera. 
Cumple todos los requisitos para ser considerada una bicicleta eléctrica:

\begin{itemize}
\item Tener dos ruedas, tal como considera El Real Decreto 2822/1998, de 23 de diciembre, que aprueba el Reglamento General de Vehículos.\cite{boe2822}
\item Contar con un motor eléctrico cuya potencia máxima no exceda los 250 \glssymbol{vatios}.
\item Alcanzar una velocidad máxima asistida de 25 \glssymbol{velocidad}.
\item El motor ha de encender con el pedaleo y apagarse al llegar a 25 \glssymbol{velocidad} o cuando el pedaleo cese.
\end{itemize}

Al cumplir todos los requisitos anteriores no es necesario seguro obligatorio, matrícula, tarjeta de inspección técnica, casco ni permiso de conducción para circular por las vías públicas. No obstante, sí es necesario que el vehículo cuente con:


\begin{itemize}
\item Doble sistema de frenado, un freno para la rueda delantera y otro para la trasera.
\item Timbre. 
\item Luces de posición. La luz delantera debe ser de color blanco y la posterior de color rojo.
\item Señalización trasera.
\end{itemize}

Todas las especificaciones numeradas están incluidas en el modelo seleccionado.

Las bicicletas eléctricas no tienen restricciones relevantes de circulación en la vía pública, pueden usar los carriles bici o la carretera en su defecto. Además, si fuese necesario pueden circular por el arcén en autovías.

\paragraph{Viabilidad del modelo}

Tal como se recoge en el plano 3 del Anexo “X”, el rango de alcance de los repartos es de 6 \glssymbol{km} poniendo como centro la localización de la sede.

Haciendo uso de los cálculos recogidos en el anexo “X” son necesarias unas 8 bicicletas eléctricas, las cuales se usarán durante todo el día y se recargarán por la noche, en el horario de cierre del negocio (24:00-9:00).

Suponiendo una revisión mensual de mantenimiento, el precio de la luz en el horario en el que se producirá la recarga de los vehículos, el coste de adquisición de los vehículos recogidos en la tabla "x". 

El coste de adquisición de los vehículos, junto y el resto de servicios requeridos por los mismos, supondría \glssymbol{euro}, tal como se expresa en la ecuación "x"

En los años siguientes, el coste en vehículos sería de aproximadamente ...\glssymbol{euro}.

De este modo, se estima que en 10 años la empresa habrá invertido ... \glssymbol{euro} en los vehículos, incluido mantenimiento, reparaciones y recargas.

La empresa ha de invertir un coste de ... bastante primer año de adquisición de los vehículos es de …. € y él anualmente se gastarían … \glssymbol{euro}.


SUPUESTA TABLA:

Precio estimado de la luz en el tramo horario 24:00-6:00 con base en los datos de los últimos meses en ese tramo horario: 0.2645 \glssymbol{euro}/\glssymbol{kilovatiohora}

Precio mantenimiento general bicicleta eléctrica en Feuvert:44,95 \glssymbol{euro}/anuales



En vista de los resultados económicos y de buena imagen de la empresa hacia el consumidor debido al método de reparto ecológico, es evidente que es una buena opción a tener en cuenta entre los distintos vehículos del mercado. No obstante, ha de considerarse que para ciertos trayectos próximos a los 6 km o que impliquen el uso de algún tramo de autovía, no es aconsejable el uso de este modelo eléctrico por seguridad del transportista y/o por exceso del tiempo de reparto deseado. Asimismo, las opciones de estacionamiento de bicicletas son limitadas, semáforos, farolas y barandillas no están permitidas, por lo que se ha de localizar un parking de bicicletas o estacionarlas del mismo modo que una moto convencional (en este último método, sería necesario también el uso de una cadena especial).

MIS fuentes:
https://www.feuvert.es/movilidad-bicis-mantenimiento/r132.html
https://tarifaluzhora.es/
https://www.almaskater.com/bicicletas-electricas-baratas/
https://www.boe.es/boe/dias/2014/05/22/pdfs/BOE-A-2014-5399.pdf
