\newpage
\section{Consideraciones preliminares}
\label{consideraciones_preliminares}

En la siguiente sección se presentan las condiciones generales que se han tomado para realizar un exhaustivo análisis de los distintos vehículos; incluyendo aspectos generales presentados en la consulta realizada por la empresa \textbf{DELICIOUS BURGER S.A.} y aspectos específicos de los distintos vehículos.

Para un correcto estudio sobre la viabilidad de incluir vehículos de gasolina o eléctricos para realizar el servicio a domicilio en el local, se deben atender una serie de consideraciones, que se estudiarán a lo largo del informe solicitado por la empresa demandante y se resumen a continuación.

\subsection{Presupuesto}
\label{consideraciones_preliminares_presupuesto}

Debido a la libertad de presupuesto por parte de la empresa, el estudio se ha enfocado en maximizar la relación calidad-precio que ofrecen los distintos modelos de vehículos, manteniendo casos realistas/factibles y que respeten las normativas y leyes correspondientes.

Para el estudio de presupuesto de cada vehículo se tendrán en consideración diversos parámetros que se describen en los siguientes apartados, tales como precio de mercado o el mantenimiento.

Por otro lado, se debe considerar el material necesario con el que se debe equipar al trabajador. Realizando un estudio sobre el equipamiento que puede necesitar, se obtiene el siguiente listado, el cual servirá para todos los modelos:

\begin{itemize}
    \item Casco de patinete \cite{cascopatineteamazon} - 41,64 \glssymbol{euro}
    \item Casco de moto \cite{cascomototeamazon} - 70,97 \glssymbol{euro}
    \item Chaleco reflectante \cite{chalecoamazon} - 14,99 \glssymbol{euro}
    \item Abrazaderas de móviles por repartidor \cite{abrazaderaamazon} - 10,99 \glssymbol{euro}
    \item Mochilas de transporte \cite{mochilaamazon} - 47,14 \glssymbol{euro}
    \item Maletero universal de moto \cite{maletero} - 93,63 \glssymbol{euro}
\end{itemize}

\subsection{Modelos y  área de reparto.}
\label{consideraciones_preliminares_area_de_reparto}

Las estimaciones proporcionadas por la empresa demandante para los recorridos son:

\begin{itemize}
    \item El rango de acción de reparto será de un radio de 6 \glssymbol{km} con respecto al local, según el \hyperref[plano:radio]{\hyperlink{plano:radio}{Plano 3}}. 
    % Referenciar al anexo del plano
    \item El viaje promedio es de 5 \glssymbol{km} de ida y vuelta, y el tiempo de reparto en el lugar de destino es de 5 minutos, de media. En secciones y anexos posteriores, se denominará caso/modelo/hipótesis 1.
    \item La atención de 2 pedidos en un mismo viaje, permite estimar una distancia de 7,5 \glssymbol{km} ida y vuelta y 10 minutos de reparto, en adelante se tratará como caso/modelo/hipótesis 2.
\end{itemize}

Las consideraciones de rango de reparto se encuentran recogidas en los \hyperref[plano:situacion]{\hyperlink{plano:situacion}{Plano 1}}  y \hyperref[plano:emplazamiento]{\hyperlink{plano:emplazamiento}{Plano 2}}, localizadas en el \refanexo{anexo_planos}, donde se puede ver el rango de alcance desde el local, con ubicación en Avenida de María Zambrano, N.º 7, C.P. 29006.

\subsection{Distribución de pedidos.}
\label{consideraciones_preliminares_tiempos}
La empresa demandante proporciona la información necesaria que recoge la densidad de pedidos semanales en distintas franjas horarias, las cuales son:

\begin{itemize}
    \item De lunes a jueves se estima que se deben atender 40 pedidos entre las 9:00 y las 20:00 horas y 50 pedidos entre las 20:00 y las 24:00.
    \item De viernes a domingo se estima que se deben atender 60 pedidos entre las 9:00 y las 20:00 horas; 120 pedidos entre las 20:00 y las 01:00.
\end{itemize}

En la \autoref{tab:distribucion de reparto} se ha representado una \textbf{distribución homogénea} de la cantidad de pedidos descritos anteriormente. Dispone de un código de colores representativos en el que se muestran las horas con menos carga de reparto y las que más tienen, indicando las situaciones en las que se produce un \gls{cuello de botella}.
\addtocounter{table}{-2}
\begin{table}[h]
\centering
\begin{tabular}{|c|c|c|c|c|c|c|c|}
\hline
\rowcolor[HTML]{DAE8FC} 
                                    & Lunes & Martes & Miércoles & Jueves & Viernes                       & Sábado                        & Domingo                       \\ \hline
\rowcolor[HTML]{9AFF99} 
\cellcolor[HTML]{DAE8FC}09:00-10:00 & 3,64  & 3,64   & 3,64      & 3,64   & \cellcolor[HTML]{34CDF9}5,45  & \cellcolor[HTML]{34CDF9}5,45  & \cellcolor[HTML]{34CDF9}5,45  \\ \hline
\rowcolor[HTML]{9AFF99} 
\cellcolor[HTML]{DAE8FC}10:00-11:00 & 3,64  & 3,64   & 3,64      & 3,64   & \cellcolor[HTML]{34CDF9}5,45  & \cellcolor[HTML]{34CDF9}5,45  & \cellcolor[HTML]{34CDF9}5,45  \\ \hline
\rowcolor[HTML]{9AFF99} 
\cellcolor[HTML]{DAE8FC}11:00-12:00 & 3,64  & 3,64   & 3,64      & 3,64   & \cellcolor[HTML]{34CDF9}5,45  & \cellcolor[HTML]{34CDF9}5,45  & \cellcolor[HTML]{34CDF9}5,45  \\ \hline
\rowcolor[HTML]{9AFF99} 
\cellcolor[HTML]{DAE8FC}12:00-13:00 & 3,64  & 3,64   & 3,64      & 3,64   & \cellcolor[HTML]{34CDF9}5,45  & \cellcolor[HTML]{34CDF9}5,45  & \cellcolor[HTML]{34CDF9}5,45  \\ \hline
\rowcolor[HTML]{9AFF99} 
\cellcolor[HTML]{DAE8FC}13:00-14:00 & 3,64  & 3,64   & 3,64      & 3,64   & \cellcolor[HTML]{34CDF9}5,45  & \cellcolor[HTML]{34CDF9}5,45  & \cellcolor[HTML]{34CDF9}5,45  \\ \hline
\rowcolor[HTML]{9AFF99} 
\cellcolor[HTML]{DAE8FC}14:00-15:00 & 3,64  & 3,64   & 3,64      & 3,64   & \cellcolor[HTML]{34CDF9}5,45  & \cellcolor[HTML]{34CDF9}5,45  & \cellcolor[HTML]{34CDF9}5,45  \\ \hline
\rowcolor[HTML]{9AFF99} 
\cellcolor[HTML]{DAE8FC}15:00-16:00 & 3,64  & 3,64   & 3,64      & 3,64   & \cellcolor[HTML]{34CDF9}5,45  & \cellcolor[HTML]{34CDF9}5,45  & \cellcolor[HTML]{34CDF9}5,45  \\ \hline
\rowcolor[HTML]{9AFF99} 
\cellcolor[HTML]{DAE8FC}16:00-17:00 & 3,64  & 3,64   & 3,64      & 3,64   & \cellcolor[HTML]{34CDF9}5,45  & \cellcolor[HTML]{34CDF9}5,45  & \cellcolor[HTML]{34CDF9}5,45  \\ \hline
\rowcolor[HTML]{9AFF99} 
\cellcolor[HTML]{DAE8FC}17:00-18:00 & 3,64  & 3,64   & 3,64      & 3,64   & \cellcolor[HTML]{34CDF9}5,45  & \cellcolor[HTML]{34CDF9}5,45  & \cellcolor[HTML]{34CDF9}5,45  \\ \hline
\rowcolor[HTML]{9AFF99} 
\cellcolor[HTML]{DAE8FC}18:00-19:00 & 3,64  & 3,64   & 3,64      & 3,64   & \cellcolor[HTML]{34CDF9}5,45  & \cellcolor[HTML]{34CDF9}5,45  & \cellcolor[HTML]{34CDF9}5,45  \\ \hline
\rowcolor[HTML]{9AFF99} 
\cellcolor[HTML]{DAE8FC}19:00-20:00 & 3,64  & 3,64   & 3,64      & 3,64   & \cellcolor[HTML]{34CDF9}5,45  & \cellcolor[HTML]{34CDF9}5,45  & \cellcolor[HTML]{34CDF9}5,45  \\ \hline
\rowcolor[HTML]{FFCC67} 
Total                               & 40    & 40     & 40        & 40     & 60                            & 60                            & 60                            \\ \hline
\rowcolor[HTML]{34CDF9} 
\cellcolor[HTML]{DAE8FC}20:00-21:00 & 12,50 & 12,50  & 12,50     & 12,50  & \cellcolor[HTML]{FE0000}24,00 & \cellcolor[HTML]{FE0000}24,00 & \cellcolor[HTML]{FE0000}24,00 \\ \hline
\rowcolor[HTML]{34CDF9} 
\cellcolor[HTML]{DAE8FC}21:00-22:00 & 12,50 & 12,50  & 12,50     & 12,50  & \cellcolor[HTML]{FE0000}24,00 & \cellcolor[HTML]{FE0000}24,00 & \cellcolor[HTML]{FE0000}24,00 \\ \hline
\rowcolor[HTML]{34CDF9} 
\cellcolor[HTML]{DAE8FC}22:00-23:00 & 12,50 & 12,50  & 12,50     & 12,50  & \cellcolor[HTML]{FE0000}24,00 & \cellcolor[HTML]{FE0000}24,00 & \cellcolor[HTML]{FE0000}24,00 \\ \hline
\rowcolor[HTML]{34CDF9} 
\cellcolor[HTML]{DAE8FC}23:00-00:00 & 12,50 & 12,50  & 12,50     & 12,50  & \cellcolor[HTML]{FE0000}24,00 & \cellcolor[HTML]{FE0000}24,00 & \cellcolor[HTML]{FE0000}24,00 \\ \hline
\rowcolor[HTML]{34CDF9} 
\cellcolor[HTML]{DAE8FC}00:00-01:00 & 0     & 0      & 0         & 0      & \cellcolor[HTML]{FE0000}24,00 & \cellcolor[HTML]{FE0000}24,00 & \cellcolor[HTML]{FE0000}24,00 \\ \hline
\rowcolor[HTML]{FFCC67} 
Total                               & 50    & 50     & 50        & 50     & 120                           & 120                           & 120                           \\ \hline
\rowcolor[HTML]{FFCC67} 
Total pedidos                       & 90    & 90     & 90        & 90     & 180                           & 180                           & 180                           \\ \hline
\end{tabular}
\caption{Distribución de reparto.\label{tab:distribucion de reparto}}
\end{table}

El código de colores es el siguiente:
\begin{itemize}
    \item \textbf{Verde:} carga de reparto inferior a 5 pedidos por hora.
    \item \textbf{Azul:} carga de reparto entre 5 y 15 pedidos por hora.
    \item \textbf{Rojo:} carga de reparto superior a 20 pedidos por hora.
\end{itemize}

Se puede observar en la \autoref{tab:distribucion de reparto}  que en la franja que abarca al turno de tarde de lunes a jueves se produce un menor volumen de encargos, permitiendo una distribución suave a lo largo de las horas, mientras que la franja de noche de viernes a domingo presentan una mayor densidad de pedidos por hora, debido al alto volumen de clientela y corto tramo horario.
% Comentar un poco más sobre el cuello de botella que se produce los findes de semana

\subsection{Coste de mantenimiento y repostaje de los vehículos}
\label{conceptos_preliminares_coste_mantenimiento_repostaje}
Todos los vehículos deben tener un coste destinado a prolongar su vida útil a lo largo de los múltiples usos anuales y deterioro físico, junto con
el desgaste en carretera.

Asimismo, se incluyen otros costes derivados que no son directamente mantenimiento, pero por simplicidad en los cálculos y según su funcionalidad, es posible agruparlos en la misma finalidad.

\subsubsection{Mantenimiento y reparación}
\label{consideraciones_preliminares_mantenimiento_reparación}
Dentro de mantenimiento y reparación del vehículo se incluyen las visitas al taller para realizar revisiones periódicas del vehículo y reparaciones de daños producidos por accidentes o deterioro de sus componentes.

Realizando un estudio de mercado, recogido en el \refanexo{anexo_precio_taller_combustible}, se han obtenido los siguientes precios medios para los vehículos de este informe:

\begin{itemize}
    \item{Motos de gasolina - 239 \glssymbol{euro}}
    \item{Motos eléctricas - 126 \glssymbol{euro}}
    \item{Patinetes eléctricos - 124,8 \glssymbol{euro}}
\end{itemize}

Además de esto, se incluye una cantidad simbólica de ahorro por tipo de vehículo para cuando el vehículo cumpla su esperanza de vida o sufra un \gls{siniestro}. Las cantidades estimadas son:

\begin{itemize}
    \item{Motos de gasolina - 100 \glssymbol{euro}}
    \item{Motos eléctricas - 100 \glssymbol{euro}}
    \item{Patinetes eléctricos - 20 \glssymbol{euro}}
\end{itemize}


\subsubsection{Seguro}
\label{consideraciones_preliminares_seguro}
Para el estudio de viabilidad de vehículos, el seguro no es de los factores más importantes que podemos encontrar, pero la empresa demandante debe considerar incluirlos para casos de accidentes o robo del vehículo.

Al igual que en el mantenimiento, se ha realizado un estudio de mercado, y el precio medio resultante es:
\begin{itemize}
    \item{Motos de gasolina y eléctricas - 265 \glssymbol{euro}}
    \item{Patinetes eléctricos - 35 \glssymbol{euro}}
\end{itemize}



\subsubsection{Precio repostaje}
\label{consideraciones_preliminares_precio_repostaje}
Por último, dentro del mantenimiento se incluye el precio por el que se compra el combustible de los vehículos a motor de combustión, o el precio de la electricidad para los vehículos eléctricos. Para el estudio realizado en este informe, se han utilizado los siguientes valores de mercado, referentes al año anterior a la fecha del informe, que se puede encontrar en \refanexo{anexo_precio_taller_combustible}:

\begin{itemize}
    \item{Gasolina - 1,38 \glssymbol{euro}/\glssymbol{litros}}
    \item{Precio luz MWh - 111,93 \glssymbol{euro}}
\end{itemize}


\subsection{Recursos humanos}
\label{consideraciones_preliminares_recursos_humanos}
En vista del tipo de estudio que solicita la empresa demandante, se ha de tener en cuenta la cantidad de personal necesario para cumplir con los pedidos necesarios por hora. Basándonos en el número de vehículos simultáneos en carretera, se puede estimar la cantidad de recursos humanos, en consecuencia, se pretenderá que el número de vehículos simultáneos sea el menor posible.

\subsection{Normativa y leyes de circulación.}
\label{consideraciones_preliminares_leyes de circulación}
Para este informe se considera que los vehículos que se van a estudiar y conductores cumplen con las normativas y leyes correspondientes de circulación recogida en la \autoref{normativa}, teniendo en cuenta las diferencias que se puedan presentar entre cada modelo y las condiciones que se han de cumplir.