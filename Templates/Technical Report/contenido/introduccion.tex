\section{Introducción}

La empresa de comida rápida \textit{DELICIOUS BURGER S.A.} (CIF: A-12345678) dispone de un local situado en Av. de María Zambrano, N.º 7, C.P. 29006, con un alto volumen de ventas.

El gobierno anuncia una serie de cambios normativos con relación a la legislación laboral en cuestiones de condiciones de subcontratación vinculadas a actividades empresariales. Por ello, junto con los movimientos estratégicos de empresas competidoras del mercado, la empresa \textit{DELICIOUS BURGER S.A.} se plantea incluir entre sus servicios el reparto a domicilio. Otras empresas competidoras han comenzado o llevan tiempo con el servicio de reparto a domicilio integrado como política de la compañía, optando en su mayoría por vehículos de tamaño reducido que permiten un cómodo y eficaz método de transporte. Por ello se ha realizado una investigación de las opciones disponibles en el mercado para realizar una primera aproximación al desarrollo del proyecto.

Se estudiarán las diferentes alternativas de vehículos, dando alternativas como: ciclomotores de gasolina, motocicletas de gasolina, motocicletas eléctricas, bicicletas eléctricas y patinetes eléctricos. Se presentará como opción para cada tipo de vehículo el modelo con mejor relación calidad precio del mercado. A continuación se compararán las necesidades de cada vehículo, así como sus costes de mantenimiento, gasto de combustible o seguro del vehículo. Se tendrán en cuenta los elementos adicionales de seguridad necesarios para cada uno, como chalecos reflectantes o cascos de motos. Con todo esto, se pretende obtener una buena aproximación del coste real de la implementación de la flota, bajo el modelo presentado, que cumple con las características requeridas por el cliente.

Se establecerán dos modelos de reparto, uno basado en realizar un pedido con un viaje medio de 5 \glssymbol{km} y otro basado en realizar dos con un viaje medio de 7,5 \glssymbol{km}. Se considerarán las diferentes ventajas y desventajas de cada vehículo y de cada modelo de reparto y se estimará cuál es la opción óptima, teniendo en cuenta la posibilidad de tomar una opción híbrida en la que se usen diferentes vehículos para maximizar la eficiencia. Para ello se tendrá en presente la viabilidad en el terreno y se proporcionará un análisis exhaustivo para que la empresa \textit{DELICIOUS BURGER S.A.} pueda valorar las diferentes propuestas y tomar una decisión con base en este informe que le permita implementar el servicio de reparto de la mejor forma posible.

Se han tenido en cuenta todas las necesidades legislativas y normativas, así como (ante la ausencia de un plano del edificio) la posibilidad de tener que construir un aparcamiento para los vehículos y la necesidad de la instalación de un circuito eléctrico para la carga de los mismos. 

Es importante tomar en consideración que ante la falta de más información, esto es un modelo aproximativo, sería necesario obtener más información por parte de la empresa para realizar un estudio mayor y más conciso.