\newpage
\section{Infraestructura}
\label{anexinfraestr}

El estudio de la infraestructura tiene diferentes variables a considerar dependiendo de cuál sea el vehículo que acabe siendo escogido, así como dependiendo de cuál sea la infraestructura con la que cuenta el local en el que se realice la instalación. Otros factores como la fecha de la obra o la duración de la misma pueden afectar al precio de esta, por lo que solo es posible realizar una estimación del coste que podría llegar a tener. Al mismo tiempo, es una variable muy importante a tener en cuenta, por lo que se intentará obtener una aproximación lo más fiel a la realidad posible. 

Ante la ausencia de planos del edificio, se debe considerar que este no dispone de ningún tipo de instalación previa adecuada para el servicio, por ello en el \refanexo{anexo:infraestructuras} se realiza un estudio y calculo de la infraestructura necesaria para las diferentes propuestas. Con esto se pretende dar una estimación lo más aproximada posible del coste de la instalación de la infraestructura necesaria para el correcto desarrollo del proyecto planteado.

El aparcamiento es el elemento fundamental de la infraestructura, es el lugar donde se van a estacionar los vehículos, ya sea entre pedidos o durante las horas no laborables. Se han tenido en cuenta las diferentes necesidades de espacio o accesorios para cada vehículo que se reflejan a continuación. 

 
\subsection{Espacio necesario para cada vehículo}
Se ha intentado reducir al máximo posible la obra necesaria, por ello se han tenido en cuenta las medidas de los vehículos y un pequeño margen de maniobra más cuatro \glssymbol{metrocuadrado} para el acceso a la via principal.

\begin{itemize}
    \item \textbf{Motocicleta Askoll eS1:} Se ha calculado un área necesaria para cada vehículo de 3330 x 1026 \glssymbol{milimetro}. \refanexo{askolles1}
    \item \textbf{KYMCO Agility Carry 50 E5:} Se ha calculado un área necesaria para cada vehículo de 3430 x 1445 \glssymbol{milimetro}. \refanexo{agilitycarry50}
    \item \textbf{KYMCO Agility Carry 125:} Se ha calculado un área necesaria para cada vehículo de 3425 x 995 \glssymbol{milimetro}. \refanexo{agilitycarry125}
    \item \textbf{Bicicleta F.Lli Schiano E-Moon:} Debido a que se trata de una bicicleta, se ha establecido la necesidad de incorporar un soporte donde poder encadenarlas, así como un candado para establecer una medida de seguridad anti-robo (que serán especificadas más adelante), el cálculo total del área se realiza para soportes de 5 bicicletas y equivale a 1357 x 2880 \glssymbol{milimetro}. \refanexo{bicicletaselectricas}
    \item \textbf{Patinete Infiniton CITYJam Pro:} Puesto que se trata de un patinete, se ha establecido la necesidad de incorporar un soporte donde poder encadenarlos, así como un candado para establecer una medida de seguridad anti-robo (especificadas más adelante), el cálculo total del área necesaria para un soporte de 8 patines con unas medidas de 1500 x 2140 \glssymbol{milimetro}. \refanexo{patineteselectrico}
 \end{itemize}

\subsection{Cimentación y Asfaltado}
Considerando la posibilidad de que sea necesario construir el espacio en el que mantener los vehículos, se ha realizado el cálculo de la cimentación y el asfaltado para cada una de las propuestas:

\begin{table}[H]
\centering
\begin{tabular}{|c|c|c|c|}
\hline
Vehículo & Área Parking (\glssymbol{metrocuadrado}) & Área Total (\glssymbol{metrocuadrado}) & Precio Total (\glssymbol{euro}) \\\hline
Askoll eS1 (Supuesto 5 km) & 30,75  & 34.75     & 4.031,00       \\\hline
Askoll eS1 (Supuesto 7.5 km)             & 23,92       & 27,98     & 3.238,72     \\\hline
KYMCO Agility Carry 50 E5                & 24,87      & 28,87    & 3.348,92     \\\hline
KYMCO Agility Carry 125                  & 17,04       & 21,04     & 2.440,64     \\\hline
F.Lli Schiano E-Moon                     & 7,82       & 11,82     & 1.371,12     \\\hline
Infinitron CITYJam Pro (Supuesto 5 km)   & 9,53        & 13,53     & 1.572,96     \\\hline
Infinitron CITYJam Pro (Supuesto 7.5 km) & 6,42        & 10,42     & 1.208,72    \\\hline
Propuesta Híbrida Moto/Patinete          & 10,04       & 14,04     & 1.628,64    \\\hline
\end{tabular}
\caption{Tabla de precios de cimentación y asfaltado.}
\end{table}


 \subsection{Propuestas de infraestructura de techado del aparcamiento}

Se presentan dos opciones para la infraestructura de techado, por un lado, una pérgola de aluminio y por otro un garaje prefabricado. Mientras que la primera opción es mucho más barata, es importante tener en cuenta que un garaje no solo añade una capa extra de protección contra el robo y las inclemencias del tiempo, sino que además suele reducir el coste del seguro del vehículo. Todo esto se traduce en una reducción de los gastos pasivos de mantenimiento.

\subsubsection{Pérgola de aluminio}
En primer lugar se presenta la opción de instalar una pérgola de aluminio la cual protegería del sol y de la lluvia tanto a los vehículos como a los posibles dispositivos de carga necesarios para estos. Por lo tanto lo hemos considerado como una propuesta muy interesante. En la siguiente tabla se presentan los precios de instalación de una pérgola para cada propuesta de vehículo. 

\begin{table}[H]
\centering
\begin{tabular}{|c|c|c|}
\hline
Vehículo                                 & Area Pergola (\glssymbol{metrocuadrado}) & Precio Total (\glssymbol{euro}) \\\hline
Askoll eS1 (Supuesto 5 km)               & 30,75     & 1.446,79     \\\hline
Askoll eS1 (Supuesto 7.5 km)             & 23,92       & 1.125,43     \\\hline
KYMCO Agility Carry 50 E5                & 24,87       & 1.170,13     \\\hline
KYMCO Agility Carry 125                  & 17,04       & 801,00      \\\hline
F.Lli Schiano E-Moon                     & 7,82        & 367,93     \\\hline
Infinitron CITYJam Pro (Supuesto 5 km)   & 9,53        & 448,39      \\\hline
Infinitron CITYJam Pro (Supuesto 7.5 km) & 6,42        & 302,06      \\\hline
Propuesta Híbrida Moto/Patinete          & 10,04       & 489,32    \\\hline
\end{tabular}
\caption{Precio de pérgola para cada propuesta de vehículo.}
\end{table}

\subsubsection{Garaje prefabricado}
Como segunda opción se cuenta con la instalación de un garaje prefabricado, el cual aportaría no solo una mayor protección contra las inclemencias del tiempo, tanto para los vehículos, como para los dispositivos de carga en caso de necesitarlos, sino que además, generalmente es un factor que reduce el coste de los seguros de vehículos. El coste de un garaje prefabricado para cada una de las propuestas se presenta a continuación.

\begin{table}[H]
\centering
\begin{tabular}{|c|c|c|}
\hline
Vehículo                                 & Area Garaje (\glssymbol{metrocuadrado}) & Precio Total (\glssymbol{euro}) \\\hline
Askoll eS1 (Supuesto 5 km)               & 30,75      & 4.099,90     \\\hline
Askoll eS1 (Supuesto 7.5 km)             & 23,92      & 3.189,25     \\\hline
KYMCO Agility Carry 50 E5                & 24,87      & 3.315,92     \\\hline
KYMCO Agility Carry 125                  & 17,04      & 2.271,94     \\\hline
F.Lli Schiano E-Moon                     & 7,82       & 1.042,64     \\\hline
Infinitron CITYJam Pro (Supuesto 5 km)   & 9,53       & 1.270,63     \\\hline
Infinitron CITYJam Pro (Supuesto 7.5 km) & 6,42       & 855,98      \\\hline
Propuesta Híbrida Moto/Patinete          & 10,04      & 1.338,63    \\\hline
\end{tabular}
\caption{Precio de garaje prefabricado para cada propuesta de vehículo.}
\end{table}

\subsection{Soporte para bicicletas o patinetes.}
En el caso de bicicletas y patinetes se ha considerado que la mejor opción para la gestión de su estacionamiento es la incorporación de estructuras de soporte, a las que se las pueda encadenar para incrementar la seguridad reduciendo la facilidad de robo. Además de que facilita la organización de los vehículos y su alineamiento con los puntos de carga. Por ello se presentan las siguientes propuestas para el encadenado de bicicletas y patinetes eléctricos.

\subsubsection{Soporte para bicicletas.}

Se ha escogido el modelo \textbf{Soporte 5 bicicletas} de la marca Btwin, por un precio de 54,99 \glssymbol{euro}, es un soporte con capacidad para 5 vehículos, y puesto que en la propuesta para bicicletas son necesarias 10, deberiamos contar don dos de ellos por un coste total de 109,98 \glssymbol{euro}.

\subsubsection{Soporte para patinetes.}

Se ha escogido el modelo \textbf{Aparcapatinetes Manuales} de la marca Adosa, por un precio de 288,61 \glssymbol{euro}. Se puede apreciar una diferencia remarcable en cuanto al precio del modelo correspondiente a las bicicletas y esto se debe a la poca oferta que hay de este producto en el mercado. Cada uno tiene capacidad para 8 vehículos, por lo que para el modelo de 5 \glssymbol{km} necesitaremos tres y para el de 7,5 \glssymbol{km} necesitaremos dos. Con un coste total respectivamente de 865,83 \glssymbol{euro} y 577,22 \glssymbol{euro}.
  
\subsection{Instalación eléctrica}
 Puesto que los vehículos eléctricos que se han considerado se cargan todos con tomas de corriente estándar necesitaremos disponer de suficientes tomas de corriente, al día con la normativa, como para en caso necesario poder cargar todos los vehículos de forma simultánea. Debido a que dependiendo de la infraestructura previa el coste puede variar mucho se ha realizado una estimación para el caso de instalarse al aire libre (bajo una pérgola) o en una zona cerrada (dentro de un garaje).
 
 \subsubsection{Instalación eléctrica de exterior}
 En el caso de una instalación eléctrica de exterior las medidas extra de seguridad necesarias suponen un sobre coste, lo que se traduce a un coste de 1.550 \glssymbol{euro} para dicha instalación.
 
 \subsubsection{Instalación eléctrica de interior}
 En el caso de una instalación eléctrica de interior los costes son inferiores, esto se traduce en un coste de 1.000 \glssymbol{euro} para la instalación.

\subsection{Opción óptima}
La opción óptima será aquella que represente una mejor inversión a largo plazo, puesto que la instalación tiene una vida útil mucho mas larga con respecto a la de un vehículo y generalmente mantiene mejor su valor. Por esto y puesto que un garaje cerrado implica no solo una reducción del desgaste del equipo (lo que se traduce en menos mantenimiento), una medida de seguridad extra y una reducción en el coste de los seguros, consideramos que la opción óptima es la de un garaje cerrado, en la que se realizará una instalación eléctrica para cargar los vehículos y se colocará un soporte para los patinetes. En caso de ser necesario se ha considerado también el coste de realizar la obra de cimentación y asfaltado del garaje.

\subsubsection{Requerimientos}
Esta opción tiene como único requisito (aparte del económico y el del espacio para la obra) la Comunicación Previa al Ayuntamiento correspondiente tal y como se presenta en la Guía Practica de Aplicación de a Declaración Responsable y Comunicación Previa en Materia de Urbanismo y Ordenación del Territorio. \cite{guiapracapdr} 




