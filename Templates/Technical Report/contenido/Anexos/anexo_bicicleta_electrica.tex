\newpage
\section{Bicicletas eléctricas}
\label{bicicleta_electrica}

\subsection{Consideraciones previas}

Las bicicletas de pedaleo deben de llevar siempre pedales y no pueden funcionar con un acelerador, por lo que es totalmente ilegal conducir por la vía pública las ``mini-bicicleta" sin pedales, según recoge el BOE y la normativa de circulación. Según la normativa vigente, excluiremos del estudio este tipo de vehículo. 

Las bicicletas eléctricas propiamente hablando son reconocidas como ``ciclo de pedaleo asistido" según el BOE N.º 124 \cite{boe124}, lo que implica que aunque tienen motor eléctrico auxiliar, no pueden ser propulsadas exclusivamente haciendo uso del mismo, por lo que son vehículos híbridos entre eléctrico y manual.

\subsection{Modelo F.Lli Schiano E-Moon}

Actualmente, una de las mejores bicicletas eléctricas del mercado calidad-precio que cumple con todas las normativas vigentes es el modelo F.Lli Schiano E-Moon, que está a la venta por 729 \glssymbol{euro}. 

Respecto a sus características, posee un motor delantero ANANDA M129F de 250 \glssymbol{vatios} de potencia y 36 \glssymbol{voltios}, que permite alcanzar velocidades de hasta 25 \glssymbol{velocidad}. Presenta una batería de litio GREENWAY YJ145 36 \glssymbol{voltios} 13 \glssymbol{amperiohora} 468 \glssymbol{vatiohora}, con una vida útil de entre 4 y 6 años y una autonomía de hasta 120 \glssymbol{km} con 6 horas de carga completa. 

Su sistema de frenado es V-Brake en las ruedas delantera y trasera, ambos neumáticos son Kenda de 26 pulgadas. El cuerpo de la bicicleta no es plegable, es de una aleación de aluminio resistente, aunque no ligera (25 \glssymbol{kg}). Consta de una caja de cambios Shimano Tourney TY21 de 7 velocidades, un timbre, una luz delantera y otra trasera. 
Cumple todos los requisitos para ser considerada una bicicleta eléctrica:

\begin{itemize}
\item Tener dos ruedas, tal como considera El Real Decreto 2822/1998, de 23 de diciembre \cite{boe2822} que aprueba el Reglamento General de Vehículos.
\item Contar con un motor eléctrico cuya potencia máxima no exceda los 250 \glssymbol{vatios}.
\item Alcanzar una velocidad máxima asistida de 25 \glssymbol{velocidad}.
\item El motor ha de encender con el pedaleo y apagarse al llegar a 25 \glssymbol{velocidad} o cuando el pedaleo cese.
\end{itemize}

Al cumplir todos los requisitos anteriores no es necesario seguro obligatorio, matrícula, tarjeta de inspección técnica, casco ni permiso de conducción para circular por las vías públicas. No obstante, sí es necesario que el vehículo cuente con:


\begin{itemize}
\item Doble sistema de frenado, un freno para la rueda delantera y otro para la trasera.
\item Timbre. 
\item Luces de posición. La luz delantera debe ser de color blanco y la posterior de color rojo.
\item Señalización trasera.
\end{itemize}

Todas las especificaciones numeradas están incluidas en el modelo seleccionado.

Las bicicletas eléctricas no tienen restricciones relevantes de circulación en la vía pública, pueden usar los carriles bici o la carretera en su defecto. Además, si fuese necesario pueden circular por el arcén en autovías.

\subsection{Viabilidad del modelo}

Tal como se recoge en el \hyperref[plano:radio]{\hyperlink{plano:radio}{Plano 3}} del \refanexo{anexo_planos}, el rango de alcance de los repartos es de 6 \glssymbol{km} poniendo como centro la localización de la sede. Haciendo uso de los cálculos recogidos en la \autoref{tab:Calculo del número de bicicletas suponiendo distancia media por reparto 5 km}, siguiendo la misma metodología del \refanexo{anexo_calculos_sobre_vehiculos}, son necesarias unas 10 bicicletas eléctricas, las cuales se usarán durante todo el día y se recargarán generalmente por la noche, en el horario de cierre del negocio (24:00-9:00).

% Tabla x
\begin{table}[H]
\centering
\begin{tabular}{l|c|c|c|c|}
\cline{2-5}
 & \begin{tabular}[c]{@{}c@{}}TRAMO \\ HORARIO 1\end{tabular} & \begin{tabular}[c]{@{}c@{}}TRAMO \\ HORARIO 2\end{tabular} & \begin{tabular}[c]{@{}c@{}}TRAMO \\ HORARIO 3\end{tabular} & \begin{tabular}[c]{@{}c@{}}TRAMO \\ HORARIO 4\end{tabular} \\ \cline{2-5} 
 & \begin{tabular}[c]{@{}c@{}}Lunes-Jueves\\ (9:00-20:00)\end{tabular} & \begin{tabular}[c]{@{}c@{}}Lunes-Jueves\\ (20:00-0:00)\end{tabular} & \begin{tabular}[c]{@{}c@{}}Viernes-Domingo\\ (9:00-20:00)\end{tabular} & \begin{tabular}[c]{@{}c@{}}Viernes-Domingo\\ (20:00-1:00)\end{tabular} \\ \hline
\multicolumn{1}{|l|}{Nº pedidos} & 40 & 50 & 60 & 120 \\ \hline
\multicolumn{1}{|l|}{\begin{tabular}[c]{@{}l@{}}Distancia total \\ a recorrer (km)\end{tabular}} & 200 & 250 & 300 & 600 \\ \hline
\multicolumn{1}{|l|}{\begin{tabular}[c]{@{}l@{}}Tiempo total de \\ reparto (min)\end{tabular}} & 660 & 240 & 660 & 300 \\ \hline
\multicolumn{1}{|l|}{\begin{tabular}[c]{@{}l@{}}Tiempo estimado por \\ pedido (min/pedido)\end{tabular}} & 20 & 20 & 20 & 20 \\ \hline
\multicolumn{1}{|l|}{\begin{tabular}[c]{@{}l@{}}Nº pedidos realizables \\ por vehículo\end{tabular}} & 33 & 12 & 33 & 15 \\ \hline
\multicolumn{1}{|l|}{\begin{tabular}[c]{@{}l@{}}Nº vehículos \\ simultáneos necesarios\end{tabular}} & 2 & 5 & 2 & 8 \\ \hline
\multicolumn{1}{|l|}{\begin{tabular}[c]{@{}l@{}}Nº de turnos \\ de vehículos\end{tabular}} & 1 & 1 & 2 & 1 \\ \hline
\multicolumn{1}{|l|}{\begin{tabular}[c]{@{}l@{}}Nº vehículos \\ totales necesarios\end{tabular}} & 2 & 5 & 4 & 8 \\ \hline
\end{tabular}
\caption{Cálculo del número de bicicletas suponiendo distancia media por reparto 5 \glssymbol{km}.}
\label{tab:Calculo del número de bicicletas suponiendo distancia media por reparto 5 km}
\end{table}


Suponiendo una revisión mensual de mantenimiento \cite{feuvertes}, el precio de la luz \cite{precio_luz_bici} en el horario en el que se producirá la recarga de los vehículos y el coste de adquisición de los vehículos \cite{precio_bici} quedan que están recogidos en la \autoref{tab:Costes economicos necesarios para el calculo del presupuesto}, el coste de adquisición de los vehículos junto y el resto de servicios requeridos por los mismos supondría 9.000,00 \glssymbol{euro} el primer año, tal como se expresa en la \autoref{eqn:presupuesto anual bicicletas}. En los años siguientes, el presupuesto para vehículos sería de aproximadamente 1.000,00 \glssymbol{euro}.


\begin{equation}
\label{eqn:presupuesto anual bicicletas}
220,06+10*(676+44,25+90)+8*114,76=9\.240,64\text{ \glssymbol{euro}}
\end{equation}


De este modo, se estima que en 10 años la empresa habrá invertido 1.800,00 \glssymbol{euro} de media anuales en los vehículos, incluido mantenimiento, reparaciones y recargas.


\begin{table}[H]
\centering
\begin{tabular}{|l|c|}
\hline
Precio medio de la luz 24:00-7:00 (€/kWh) & 0,2645 \\ \hline
Capacidad batería vehículo (\glssymbol{kilovatiohora}) & 0,25 \\ \hline
Presupuesto anual en recargas (\glssymbol{euro}) & 220,06 \\ \hline
Precio unitario/vehículo (\glssymbol{euro}) & 676,00 \\ \hline
Precio unitario equipamiento (\glssymbol{euro}) & 114,76 \\ \hline
Precio unitario mantenimiento bicicleta eléctrica en Feuvert (€/año) & 44,25 \\ \hline
Reserva económica unitaria para sustituciones (€/año) & 90,00 \\ \hline
\end{tabular}
\caption{Costes económicos necesarios para el cálculo del presupuesto.}
\label{tab:Costes economicos necesarios para el calculo del presupuesto}
\end{table}

En vista de los resultados económicos y de buena imagen de la empresa hacia el consumidor debido al método de reparto ecológico, es evidente que es una buena opción a tener en cuenta entre los distintos vehículos del mercado. No obstante, ha de considerarse que para ciertos trayectos próximos a los 6 km o que impliquen el uso de algún tramo de autovía, no es aconsejable el uso de este modelo eléctrico por seguridad del transportista y/o por exceso del tiempo de reparto deseado. 

Asimismo, las opciones de estacionamiento de bicicletas son limitadas, semáforos, farolas y barandillas no están permitidas, por lo que se ha de localizar un aparcamiento de bicicletas o estacionarlas del mismo modo que una moto convencional (en este último método, sería necesario también el uso de una cadena especial).

\newpage
% \addcontentsline{toc}{section}{Referencias}
% \section*{Referencias}
% \label{referencias_nucleo}
% \makeatletter
% \def\@bibitem#1{\item\if@filesw \immediate\write\@auxout
%   {\string\bibcite{#1}{A\the\value{\@listctr}}}\fi\ignorespaces}
% \def\@biblabel#1{[A{#1}]}
% \makeatother
% \printbibheading[title={Referencias},heading=bibintoc]
\nocite{*}
\newrefcontext[labelprefix=\thesection.]
\printbibheading[title={Referencias},heading=subbibintoc]
\printbibliography[heading=none,resetnumbers=true,keyword=bicielec]
\newpage
