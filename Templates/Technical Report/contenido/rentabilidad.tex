\newpage
\section{Rentabilidad}
\label{Rentabilidad}

Tomando en consideración las propuestas estudiadas, se puede realizar una estimación sobre la rentabilidad que ofrecen a largo plazo a la empresa. Para ello, tomaremos los costes iniciales y anuales de los vehículos, junto con el precio de la instalación. Tras tener el conjunto de inversiones necesarias, realizaremos una simulación de ganancias y perdidas obtenidas con el precio medio de pedido y el salario mínimo actuales en España. Para la simplificación del estudio e investigar el propósito de este apartado, se realizará una comparación entre la opción de gasolina, con la moto KYMCO Agility Carry 125, y la opción híbrida de motos eléctricas y patinetes, la Askoll eS1 y la Infiniton CITYJam Pro.

Los modelos de estudio son los proporcionados por la empresa, siendo el modelo 1 el de un reparto a 5 \glssymbol{km} de media, y el modelo 2 el de dos repartos a 7,5 \glssymbol{km} de media.

Los datos necesarios para la rentabilidad se recogen en la \autoref{tab: datos de rentabilidad}.

\begin{table}[H]
\centering
\begin{tabular}{ccc}
\hline
\multicolumn{1}{|c|}{}                          & \multicolumn{1}{c|}{Ganancia/Pérdida para 5 km (€)} & \multicolumn{1}{c|}{Ganancia/Pérdida para 7,5km (€)} \\ \hline
\multicolumn{1}{|c|}{Coste fijo moto}           & \multicolumn{1}{c|}{- 18.058,14}                    & \multicolumn{1}{c|}{- 17.591,70}                     \\ \hline
\multicolumn{1}{|c|}{Coste anual moto}          & \multicolumn{1}{c|}{- 5.864,39}                     & \multicolumn{1}{c|}{- 5.394,95}                      \\ \hline
\multicolumn{1}{|c|}{Coste instalación garajes} & \multicolumn{1}{c|}{- 2.271,94}                     & \multicolumn{1}{c|}{- 2.271,94}                      \\ \hline
\multicolumn{1}{l}{}                            & \multicolumn{1}{l}{}                                & \multicolumn{1}{l}{}                                 \\ \hline
\multicolumn{1}{|c|}{}                          & \multicolumn{1}{c|}{Ganancia/Pérdida para 5 km (€)} & \multicolumn{1}{c|}{Ganancia/Pérdida para 7,5km (€)} \\ \hline
\multicolumn{1}{|c|}{Coste fijo moto/patinete}  & \multicolumn{1}{c|}{- 16.536,67}                    & \multicolumn{1}{c|}{- 15.362,72}                     \\ \hline
\multicolumn{1}{|c|}{Coste anual moto/patinete} & \multicolumn{1}{c|}{- 4.205,91}                     & \multicolumn{1}{c|}{- 3.785,20}                      \\ \hline
\multicolumn{1}{|c|}{Coste instalación garajes} & \multicolumn{1}{c|}{- 1.338,63}                     & \multicolumn{1}{c|}{- 1.338,63}                      \\ \hline
\multicolumn{1}{l}{}                            & \multicolumn{1}{l}{}                                & \multicolumn{1}{l}{}                                 \\ \cline{1-2}
\multicolumn{1}{|c|}{}                          & \multicolumn{1}{c|}{Ganancia/Pérdida (€)}           &                                                      \\ \cline{1-2}
\multicolumn{1}{|c|}{Salario mínimo}            & \multicolumn{1}{c|}{-1.108,3}                       &                                                      \\ \cline{1-2}
\multicolumn{1}{|c|}{Salario mínimo anual}      & \multicolumn{1}{c|}{- 13.299,6}                     & \multicolumn{1}{l}{}                                 \\ \cline{1-2}
\multicolumn{1}{|c|}{Ganancia media por pedido} & \multicolumn{1}{c|}{+ 15,00}                        &                                                      \\ \cline{1-2}
\multicolumn{1}{|l|}{Ganancia por pedido anual} & \multicolumn{1}{c|}{+ 46.800}                       & \multicolumn{1}{l}{}                                 \\ \cline{1-2}
\end{tabular}
\caption{Datos de rentabilidad}
\label{tab: datos de rentabilidad}
\end{table}

Realizando la contabilidad para los dos modelos de reparto, en los vehículos de gasolina y los eléctricos, y añadiéndole las ganancias, y el salario mínimo para cinco trabajadores en gasolina, y once y nueve repartidores para eléctricos, con los dos modelos de reparto respectivamente. Los datos que se obtienen a lo largo de cinco años son los mostrados en la \autoref{tab: beneficios a lo largo de 5 años}.

\begin{table}[H]
\centering
\resizebox{\textwidth}{!}{
\begin{tabular}{|c|c|c|c|c|c|}
\hline
                                 & Año 1 (€)  & Año 2 (€)  & Año 3 (€)  & Año 4 (€)  & Año 5 (€)  \\ \hline
KYMCO Agility Carry 125 modelo 1 & 609.307,53 & 629.637,61 & 629.637,61 & 629.637,61 & 629.637,61 \\ \hline
KYMCO Agility Carry 125 modelo 2 & 610.243,41 & 630.107,05 & 630.107,05 & 630.107,05 & 630.107,05 \\ \hline
Askoll eS1 e Infiniton CITYJam Pro modelo 1 & 533623,19  & 551.498,49 & 551.498,49 & 551.498,49 & 551.498,49 \\ \hline
Askoll eS1 e Infiniton CITYJam Pro modelo 2 & 561.817,05 & 578518,40  & 578518,40  & 578518,40  & 578518,40  \\ \hline
\end{tabular}}
\caption{Beneficios a lo largo de 5 años}
\label{tab: beneficios a lo largo de 5 años}
\end{table}

Viendo los datos obtenidos, se comprueba que se producen beneficios todos los años, incluso el primero, con los costes de compra de vehículos e instalaciones. Se observa que escogiendo la alternativa de gasolina se obtendrían más beneficios, pero a nivel de marketing y apariencia al público se favorece la elección de los modelos eléctricos.

También hay que tener en cuenta que los resultados obtenidos son en base a un precio estimado, y a que se cumplan semanalmente el número de pedidos que se indican en los requisitos solicitados por la empresa, pero la diferencia entre lo ideal y real seguirá resultando en beneficios en brutos, donde la empresa deberá descontar lo correspondiente a gastos de productos para cocinar los pedidos, impuestos, alquiler de local, entre otras cosas.

%Ingreso fijo y anual


%Coste infraestructura - garaje cerrado = 1.338,63 \glssymbol{euro}
%Coste instalación eléctrica = 1.000 \glssymbol{euro}
%%Aparcapatinetes =  577,22 \glssymbol{euro}

 


%salario mínimo (Pérdida) 1.108,3 €
%Precio medio comida (Ganancia)