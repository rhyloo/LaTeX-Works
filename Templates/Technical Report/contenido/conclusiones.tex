\newpage
\addcontentsline{toc}{section}{Conclusiones}
\section*{Conclusiones}
\label{conclusiones}

A lo largo de este Informe Técnico se han estudiado una variedad de alternativas para cumplir con los requisitos iniciales del problema. En base a esos requisitos, se han estudiado los distintos modelos de vehículos existentes en el mercado, su efecto económico a la empresa y al medio ambiente, la infraestructura necesaria para acoger a la flota de vehículos final.

Es por ello por lo que tras terminar el estudio se ha concluido que, a la pregunta que se plantea, sobre si se debería adquirir equipos de transporte en un formato 100\% eléctrico en lugar de vehículos con motores de gasolina para ofrecer el servicio de reparto, sí se debería.

El primer motivo por lo que se recomienda una flota de transporte eléctrico es la publicidad y la opinión pública positiva que genera a largo plazo, dado que el mundo está poco a poco cambiando su estilo de vida por uno más ecológico para el medio ambiente y la Tierra.

El segundo motivo son los costes que generan adquirir el equipo a corto y largo plazo. A pesar de ser una tecnología relativamente nueva en el mercado, con el tiempo se va reduciendo el coste de producción, implicando un precio de mercado menor. Además de esto, tras el estudio realizado se observa que tanto en equipo como en medio de combustibles es más económico que los combustibles fósiles.

Una vez contestada la cuestión, se profundiza en el estudio de distintos modelos, y de estos un análisis más profundo de los que más destacan de cada categoría.

Con este análisis se concluye que, a pesar de no ser la más económica, la mejor alternativa para la empresa es adquirir una combinación de patinetes y motos eléctricas, los cuáles son la Infiniton CITYJam Pro y la Askoll eS1 respectivamente.

Los motivos por lo que se opta por este conjunto de vehículos, aparte de los mencionados previamente, son los modelos de reparto proporcionados y el radio de reparto que ofrece el local. Los patinetes electrónicos son los más baratos, incluso si se requieres del uso de múltiples al día, pero su desventaja se presenta cuando debe realizar un reparto a largas distancias, como Churriana, donde no puede ir por autovías. Por otro lado, una moto, a pesar de poder ir a largas distancias, no sale tan rentable para recorridos cortos.

La unión de motos y patinetes permiten aprovechar las ventajas de ambos modelos y subsanar las debilidades que presentan, y la instalación necesaria para poder recargarlos no es excesivo, comparado a otros vehículos de mayor tamaño, como los coches eléctricos.

Así pues, una vez más, la recomendación que se ofrece con toda la información obtenida es la adquisición de motos y patinetes eléctricos.

En cuanto a la instalación necesaria, la recomendación es la construcción de un garaje cerrado, ya que ofrece seguridad a la hora de almacenar los vehículos, y permite prolongar su vida útil frente al desgaste de las condiciones meteorológicas.

\newpage