\section*{}
\begin{center}
    \textbf{\noindent ARTÍCULO 335.2 DE LA LEY 1/2000, DE 7 DE ENERO, DE ENJUICIAMIENTO CIVIL (LEC)}
\end{center}

``El perito declara, bajo juramento de decir verdad, que ha actuado y, en su caso, actuará con la mayor objetividad posible, tomando en consideración tanto lo que se pueda favorecer como lo que sea susceptible de causar perjuicio a cualquiera de las partes, y que conoce las sanciones penales en las que podría incurrir si incumpliere su deber como perito.''
\vspace{0.2\textheight}
\begin{center}
    \textbf{\noindent ARTÍCULO 343 DE LA LEY 1/2000, DE 7 DE ENERO, DE ENJUICIAMIENTO CIVIL (LEC)}
\end{center}

``El perito que suscribe el dictamen, manifiesta:
\begin{itemize}
    \item No ser cónyuge o pariente por consanguinidad o afinidad, dentro del cuarto grado civil de una de las partes o de sus abogados o procuradores.
    \item No tener interés directo o indirecto en el asunto o en otro semejante.
    \item No estar o haber estado en situación de dependencia o de comunidad o contraposición de intereses con alguna de las partes o con sus abogados.
    \item No tener amistad íntima o enemistad con cualquiera de las partes o sus procuradores y abogados.
    \item No creer que exista ninguna otra circunstancia que le haga desmerecer en el concepto profesional.''
\end{itemize}

\newpage
