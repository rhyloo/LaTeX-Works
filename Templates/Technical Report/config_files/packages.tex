\makeatletter
\def\identificador#1{\def\@identificador{#1}}
\def\identificadorx{\@identificador}% Display title

\def\seguridad#1{\def\@seguridad{#1}}
\def\seguridadx{\@seguridad}% Display title

\def\titulo#1{\def\@titulo{#1}}
\def\titulox{\@titulo}% Display title

\def\subtitulo#1{\def\@subtitulo{#1}}
\def\subtitulox{\@subtitulo}% Display title

\def\tipo#1{\def\@tipo{#1}}
\def\tipox{\@tipo}% Display title

\def\organizacion#1{\def\@organizacion{#1}}
\def\organizacionx{\@organizacion}% Display title

% \def\organizacion#1{\def\@organizacion{#1}}
% \def\organizacionx{\@organizacion}% Display title

\def\asociacion#1{\def\@asociacion{#1}}
\def\asociacionx{\@asociacion}% Display title

\newcommand*{\logo}[2][scale=0.5]{%
  \gdef\@logo@params{#1}%
  \gdef\@logo{#2}%
}
\newcommand*{\@logo@params}{}
\newcommand*{\@logo}{}
\def\logox{\expandafter\includegraphics\expandafter[\@logo@params]{\@logo}}

\def\autor#1{\def\@autor{#1}}
\def\autorx{\@autor}% Display title

\def\tutor#1{\def\@tutor{#1}}
\def\tutorx{\@tutor}% Display title

\def\fecha#1{\def\@fecha{#1}}
\def\fechax{\@fecha}% Display title

 \makeatother

\setlength{\parskip}{1em}

\makeatletter
    \renewcommand\paragraph{%
        \@startsection{paragraph}{4}{0mm}%
           {-\baselineskip}%
           {.5\baselineskip}%
           {\normalfont\normalsize\bfseries}}
    \makeatother

%  \usepackage[utf8]{inputenc}
 %%%%%%%%%%%%%%%%%%%%%%%%
% DOCUMENTO EN ESPAÑOL
%%%%%%%%%%%%%%%%%%%%%%%%
\usepackage{siunitx}
\usepackage{makecell}
\usepackage{tabularx,ragged2e}
\newcolumntype{C}{>{\Centering\arraybackslash}X} % centered "X" column
\usepackage{multirow}
\usepackage[table,xcdraw]{xcolor}
% \usepackage[titletoc,title,toc,page]{appendix}
\usepackage[titletoc,title]{appendix}
% \appto\appendix{\addtocontents{toc}{\protect\setcounter{tocdepth}{4}}}
\renewcommand\appendixtocname{Anexos}
\usepackage[spanish]{babel}
% \usepackage{polyglossia}
\usepackage{chngcntr}
\usepackage[pagecolor=none]{pagecolor}

\addto\captionsspanish{%
    % \renewcommand\appendixtocname{Anexos}
    % \renewcommand{\appendixname}{Anexo}
    % \renewcommand{\appendixpagename}{Anexos}
    \renewcommand{\appendixname}{Anexo}
    \renewcommand{\tablename}{Tabla}
	\renewcommand{\listtablename}{Índice de tablas} 
	\renewcommand{\lstlistingname}{Código}
	\renewcommand{\lstlistlistingname}{Índice de \lstlistingname s}
	\renewcommand{\glossaryname}{Glosario}
	\renewcommand{\acronymname}{Acrónimos}
	\renewcommand{\bibname}{Bibliografía}%
}
\addto\extrasspanish{%
  \renewcommand*{\appendixautorefname}{Anexo}%
}


% \addto\extrasspanish{%
% %   \def\figurename{Figura}%
% }
\usepackage[hyphens]{url} 
\def\UrlBreaks{\do\/\do-}
 \usepackage{hyperref}
 \hypersetup{
    hidelinks,
    breaklinks,
    colorlinks=false,
    linkcolor=blue,
    pdftitle={IT-PI-04},
    pdfpagemode=FullScreen,
    }
 \usepackage{pdfpages}
 \newcounter{pdfpage}
%%%%%%%%%%%%%%%%%%%%%%%% 
% GLOSARIOS
%%%%%%%%%%%%%%%%%%%%%%%%
\usepackage[acronym,nonumberlist]{glossaries}
\usepackage{glossary-superragged}
\newglossarystyle{modsuper}{%
  \setglossarystyle{super}%
  \renewcommand{\glsgroupskip}{}
}
\renewcommand{\glsnamefont}[1]{\textbf{#1}}
\usepackage{everyshi}
\usepackage{multirow}
\usepackage{scrlayer-scrpage}
\setkomafont{pageheadfoot}{} % define font style for page head foot
\clearpairofpagestyles % remove the defaults
% \cfoot{- \pagemark \ -} % put the page number in the outer footer
\usepackage{titletoc}
\usepackage{lastpage}
\ihead{\seguridadx}
\ohead{\identificadorx}
\newcounter{apppage}
\cfoot{- \thepage /\pageref{lastpage} -}

\usepackage{lipsum}
\usepackage{soul}
\usepackage[official]{eurosym}
\usepackage[shortlabels]{enumitem}
\usepackage{float}
\usepackage{scrhack} % Previene algunos errores

\usepackage{lscape}
% Ancho de la zona para comentarios en el margen. (modificado para todonotes)

\usepackage{listings}
\usepackage[textsize=tiny,spanish,shadow,textwidth=2cm]{todonotes}




% Paquetes
\usepackage[
  inner	=	3.0cm, % Margen interior
  outer	=	2.5cm, % Margen exterior
  top	=	2.5cm, % Margen superior
  bottom=	2.5cm, % Margen inferior
  includeheadfoot, % Incluye cabecera y pie de página en los márgenes
]{geometry}
\newgeometry{ignoreall,top=2cm,outer=2cm,inner=2cm}
\thispagestyle{empty}
\usepackage{graphicx}
\usepackage{calc}
\usepackage{xstring}
\usepackage{setspace}
\usepackage{ifthen}
\usepackage{tikz}
\usetikzlibrary{calc}
\usetikzlibrary{positioning}
\usetikzlibrary{fit}
\usetikzlibrary{shapes}
\usetikzlibrary{arrows}
\usepackage{eso-pic}
%\usepackage{cite}
\usepackage{breakcites}
%%%%%%%%%%%%%%%%%%%%%%%%
% TEXTO
%%%%%%%%%%%%%%%%%%%%%%%%
\usepackage{fontspec}
% Paquete para poder modificar las fuente de texto
\usepackage{xltxtra}
% Cualquier tamaño de texto. Uso: {\fontsize{100pt}{120pt}\selectfont tutexto}
\usepackage{anyfontsize}
% Para modificar parametros del texto.
\usepackage{setspace}
% Paquete para posicionar bloques de texto
\usepackage{textpos}
% Paquete para realizar cajas de texto.
% Uso: \begin{mdframed}[linecolor=red!100!black] tutexto \end{mdframed}
\usepackage{framed,mdframed}
% Para subrayar. Uso: \hlc[tucolor]{tutexto}
\newcommand{\hlc}[2][yellow]{ {\sethlcolor{#1} \hl{#2}} }
\newfontfamily\FuenteTitulo{HelveticaLTStd-Cond}[Path=./include/fuentes/]
% Helvetica. Uso: {\FuentePortada tutexto}
\newfontfamily\FuentePortada{Helvetica}[Path=./include/fuentes/]


\StrLen{\@titulox}[\longitudtitulo] % Cuenta los caracteres título
\definecolor{fondo}{RGB}{32,2,116}
\newcommand{\colorfondo}{white}



% Tamaño por defecto de la fuente de texto para:
\def\FuenteTamano{55pt}	% Tamaño para el título del trabajo
\def\interlinportada{5.0} % Interlineado por defecto para el título
\def\TamTrabajo{20pt} 	% Tamaño para el tipo de trabajo (grado o máster)
\def\TamTrabajoIn{20pt} 	% Tamaño para el salto de línea después de tipo de trabajo
\def\TamOtros{15pt} 	% Tamaño para datos personales y fecha
\def\TamOtrosIn{1pt} 	% Tamaño para los saltos de línea en la info personal


% Comprueba la longitud del título y según sea este determina unos valores nuevos
\ifthenelse{\longitudtitulo > 180}{
\def\FuenteTamano{35pt}		% Si es mayor a 180 caracteres tamaño de fuente 35pt
\def\interlinportada{3.5}} 	% Establece nuevo interlineado
{\ifthenelse{\longitudtitulo > 140}{
\def\FuenteTamano{40pt}		% Si es mayor a 140 caracteres tamaño de fuente 40pt
\def\interlinportada{4.0}} 	% Establece nuevo interlineado
{\ifthenelse{\longitudtitulo > 120}{
\def\FuenteTamano{50pt}		% Si es mayor a 120 caracteres tamaño de fuente 50pt
\def\interlinportada{4.5}} 	% Establece nuevo interlineado
{} % Si no, no modifica el tamaño
} }

\renewcommand{\appendixpagename}{\topskip0pt
\vspace*{\fill}
\centering
Anexos
\label{Anexos}
\vspace*{\fill}}


\def\sectionautorefname{Section}
\def\subsectionautorefname{Section}
\def\appendixautorefname{Appendix}

% begin appendix autoref patch [\autoref subsections in appendix](https://tex.stackexchange.com/questions/149807/autoref-subsections-in-appendix)
\usepackage{etoolbox}
\pretocmd\section{\cleardoublepage}{}%
  {\errmessage{Patching \noexpand\section failed}}
  
\makeatletter
\patchcmd{\hyper@makecurrent}{%
    \ifx\Hy@param\Hy@chapterstring
        \let\Hy@param\Hy@chapapp
    \fi
}{%
    \iftoggle{inappendix}{%true-branch
        % list the names of all sectioning counters here
        \@checkappendixparam{chapter}%
        \@checkappendixparam{section}%
        \@checkappendixparam{subsection}%
        \@checkappendixparam{subsubsection}%
        \@checkappendixparam{paragraph}%
        \@checkappendixparam{subparagraph}%
    }{}%
}{}{\errmessage{failed to patch}}

\newcommand*{\@checkappendixparam}[1]{%
    \def\@checkappendixparamtmp{#1}%
    \ifx\Hy@param\@checkappendixparamtmp
        \let\Hy@param\Hy@appendixstring
    \fi
}
\makeatletter

\newtoggle{inappendix}
\togglefalse{inappendix}

\apptocmd{\appendix}{\toggletrue{inappendix}}{}{\errmessage{failed to patch}}
\apptocmd{\subappendices}{\toggletrue{inappendix}}{}{\errmessage{failed to patch}}
% end appendix autoref patch

\newcommand{\refanexo}[1]{\hyperref[#1]{Anexo~\ref{#1}}}
%\refanexo{anexo_xxx}
%https://github.com/abntex/abntex2/issues/76

\usepackage[titles]{tocloft}

% \appto\appendix{%
%   \setcounter{secnumdepth}{3}
%   \addtocontents{toc}{%
%     \unexpanded{\unexpanded{%
%       \cftpagenumbersoff{subsection}%
%       \cftpagenumbersoff{section}%
%     %   \setlength\cftsecindent{\cftsecnumwidth}%
%     }}%
%   }%
% }

\setlength{\glsdescwidth}{15cm}
\setlength{\headheight}{2em}
% \setlength{\parindent}{0em}
\setlength{\footheight}{4em}
\renewcommand{\baselinestretch}{1.1}

\newglossary[slg]{symbolslist}{syi}{syg}{Symbolslist} % create add. symbolslist


\glsaddkey{unit}{\glsentrytext{\glslabel}}{\glsentryunit}{\GLsentryunit}{\glsunit}{\Glsunit}{\GLSunit}



\newglossarystyle{symbunitlong}{%
\setglossarystyle{long3col}% base this style on the list style
\renewenvironment{theglossary}{% Change the table type --> 3 columns
  \begin{longtable}{lp{0.6\glsdescwidth}>{\centering\arraybackslash}p{2cm}}}%
  {\end{longtable}}%
%
\renewcommand*{\glossaryheader}{%  Change the table header
  \bfseries Símbolo & \bfseries Descripción\\
  \hline
  \endhead}
\renewcommand*{\glossentry}[2]{%  Change the displayed items
% Description
\glsunit{##1} & \glossentrydesc{##1}  \tabularnewline
}
}

\usepackage[sorting=none,citestyle=numeric-comp,defernumbers=true,backend=biber]{biblatex}

\usepackage{csquotes}
\addbibresource{bibliografia.bib}

% % Count total number of entries in each refsection
% \AtDataInput{%
%   \csnumgdef{entrycount:\therefsection}{%
%     \csuse{entrycount:\therefsection}+1}}

% % Print the labelnumber as the total number of entries in the
% % current refsection, minus the actual labelnumber, plus one
% \DeclareFieldFormat{labelnumber}{\mkbibdesc{#1}}
% \newrobustcmd*{\mkbibdesc}[1]{%
%   \number\numexpr\csuse{entrycount:\therefsection}+1-#1\relax}

% \DeclareFieldFormat{labelnumberwidth}{#1.}

\newcounter{bibIndex}

\makeatletter
\newcommand{\changeBibPrefix}[1]{%
  \def\@bibitem##1{%
    \@skiphyperreftrue\H@item\@skiphyperreffalse
    \Hy@raisedlink{%
      \hyper@anchorstart{cite.##1\@extra@b@citeb}\relax\hyper@anchorend
    }%
    \if@filesw
    \begingroup
      \let\protect\noexpand
      \immediate\write\@auxout{%
        \string\bibcite{##1}{#1\the\value{\@listctr}}%
      }%
    \endgroup
    \fi
    \ignorespaces
  }
  \def\@biblabel##1{[#1##1]}
}
\makeatother