%%%%%%%%%%%%%%%%%%%%%%%%%%%%%%%%%%%%%%%%%%%%%%%%%%%%%%%%%%%%%%%%%%%%%%%%
%%																	  %%
%%		         	           PAQUETES       					      %%
%%																	  %%
%%%%%%%%%%%%%%%%%%%%%%%%%%%%%%%%%%%%%%%%%%%%%%%%%%%%%%%%%%%%%%%%%%%%%%%%

%%Para las tildes.
\usepackage[utf8]{inputenc}		
\DeclareUnicodeCharacter{2212}{-}

\usepackage{lipsum}
%%Para la codificación de las fuentes de salida, T1 para comillas y acentos.
\usepackage[T1]{fontenc}		

%%Evita la fuente pixelada tras la carga de fontenc.
\usepackage{lmodern}		    

%%Para la correcta división silábica en español.
\usepackage[spanish]{babel}		

%%Para mejoras varias y tener el comando \text{},
%%para tener comandos como \Cap, \Cup, \leadsto Y para tener \mathbb, \mathfrak
\usepackage{amsmath,amssymb,amsfonts}		
\usepackage{txfonts}

%%Para poder hacer \mathscr{F}
\usepackage{mathrsfs}		

%%Para poder usar el comando afterpage.
%\usepackage{afterpage}		

%%Para dividir el documento en 2 columnas.
%\usepackage{multicol}		
\usepackage[siunitx, RPvoltages]{circuitikz}
\usepackage{catchfilebetweentags} 
%%Mejores tablas.
%\usepackage{tabu}		
\usepackage{enumerate}
%%Para modificar opciones de las listas.
% \usepackage{enumitem}		
% 	\setlist[enumerate]{labelindent=!, leftmargin=*, label=(\roman*),widest=VIII, align=left}
	
%%Para tener un índice dinámico y bonito.
\usepackage[hidelinks]{hyperref}		
\hypersetup{
    colorlinks = true,
    linkcolor = blue,
    urlcolor= orange
}
%%Para evitar un error con las opciones de xcolor.
    \PassOptionsToPackage{usenames,dvipsnames}{xcolor}		


%% Permite usar directrices como [H]
\usepackage{float}

\usepackage{esvect}
%%Para usar colores. Usamos \textcolor{Turquoise}{#1} para colorear #1.
\usepackage{xcolor}		

%%Para cargar imágenes 
    %Usaremos \includegraphics[scale=•]{•}.
\usepackage{graphicx}		

%%Coloured boxes, for LaTeX examples and theorems, etc
\usepackage{tcolorbox}

%%Para realizar gráficas y dibujos
\usepackage{tikz}		
%%Para poder usar flechas y 3D sin problema.    
    \usetikzlibrary{babel,3d,arrows.meta, arrows}

%%Para la realización de gráficas dentro de Tikz.
\usepackage{pgfplots}		
    %%Ancho(predet.) de las gráficas de pgfplots.
% 	\pgfplotsset{width=10cm,compat=1.9}

\usepackage{pgf}
 \pgfplotsset{compat=1.15}

%%Para modificar los márgenes por defecto del documento.
\usepackage[a4paper]{geometry}		
    %%Para especificar dichos márgenes.
	\geometry{hmargin={2.3cm,2.3cm},height=24.5cm}		
	
%%Para personalizar encabezados y pies de página
%%se distingue entre Article y Book, cuidado con los comandos.
\usepackage{fancyhdr}		
    \pagestyle{fancy}

%%Paquete útil para generar un texto y hacer modificaciones
\usepackage{lipsum}

%%Paquete de asymptote
\usepackage[inline]{asymptote}

%%Paquuete espacio entre líneas
\usepackage{setspace}
    \setlength{\parskip}{1em}              %% Espaciado entre párrafos
    \setlength{\parindent}{0em}             %% Sangría
    \renewcommand{\baselinestretch}{1.2}    %% Interlineado						

%%No sé qué hace (aún)
\usepackage{titlesec}
\titlespacing*{\section}{0em}{0em}{0em}

%% Soluciona la justificación de la letra monoespacida.
\usepackage{everysel}
    \renewcommand*\familydefault{\ttdefault}
    \EverySelectfont{%
    \fontdimen2\font=0.4em  % interword space
    \fontdimen3\font=0.2em  % interword stretch
    \fontdimen4\font=0.1em  % interword shrink
    \fontdimen7\font=0.1em  % extra space
    \hyphenchar\font=`\-    % to allow hyphenation
}
%%Imágenes en svg
\usepackage{svg}

%%%%%%%%%%%%%%%%%%%%%%%%%%%%%%%%%%%%%%%%%%%%%%%%%%%%%%%%%%%%%%%%%%%%%%%%
%%																	  %%
%%		         	      Configuraciones      					      %%
%%																	  %%
%%%%%%%%%%%%%%%%%%%%%%%%%%%%%%%%%%%%%%%%%%%%%%%%%%%%%%%%%%%%%%%%%%%%%%%%

\widowpenalty=10000		%Para evitar líneas viudas.
\clubpenalty=10000		%Para evitar líneas huérfanas.

\makeatletter
    \@beginparpenalty=10000
\makeatother


%%Grosor de la línea del header (encabezado).
\renewcommand{\headrulewidth}{2pt}	
\usepackage{pdfpages}
\usepackage{pdflscape}
%%Alto del encabezado
\setlength{\headheight}{15pt}
    \fancyhead[LO]{\href{https://t.me/Rhyloo}{\footnotesize Rhyloo}}
 	\fancyfoot[RO]{\thepage}
    \fancyhead[RE]{\href{https://t.me/Rhyloo}{\footnotesize Rhyloo}}
    \fancyfoot[C]{}
 	\fancyfoot[LE]{\thepage}

%%Permite cambiar la palabra Capítulo por Tema, Clase Magistral, etc...
    % Estructura
        %\addto\captions<language>{\renewcommand{\chaptername}{<new name>}}
    % Ejemplo
         \addto\captionsspanish{\renewcommand{\chaptername}{Relación de problemas}}
