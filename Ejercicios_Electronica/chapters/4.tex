\header{chapter4}
\chapter{}
\label{chapter4}

\mkexercise{Calcular anal\'iticamente y mediante simulaci\'on el punto de
  trabajo $Q$($V_d$.$I_d$) del diodo D1N4148 tal como muestra el
  circuito de la figura y a temperaturas de 27$^\circ$ y 80$^\circ$
  cent\'igrados.}
\mkexercise{Realizar el problema anterior usando el modelo clásico del diodo
  en el primer cuadrante.}
\mkexercise{Mediante simulación, trazar la curva de característica
  tensión-corriente del diodo D1N4148.}
\mkexercise{Mediante simulación, trazar la curva característica
  tensión-corriente del diodo zener D1N4733 que tiene su codo zener en
  5.1 voltios.}
\mkexercise{Realizar mediante simulación un análisis temporal del
  circuito de la figura en el que aparece el diodo D1N4148 excitado
  con una señal senoidal de una amplitud pico a pico de 30 voltios y
  una frecuencia de 50 Hz.}
\mkexercise{Calcular con el simulador y analíticamente, la curvaa de
  transferencia (o característica de transferencia) del siguiente
  circuito en el rango de entrada $-5 V \leq V_{in} \leq 5 V$. Nota:
  Considerar una resistencia $R_d$ del diodo para el modelo
  linealizado en directa de 10 $\Omega$.}
\mkexercise{Utilizando el procedimiento de análisis por estados del
  diodo, analizar el circuito de la figura utilizando del modelo de
  diodo ideal. Verificarlo mediante simulación.}
\mkexercise{El circuito de la figura muestra un limitador paralelo
  simétrico. Dibujar su curva de transferencia $(V_{in},V_{out})$ de
  forma analítica y por simulación.}
\mkexercise{El circuito de la figura muestra el circuito denominado
  ``rectificador de diodos en puente'' utilizado en circuitos de
  conversión AC/DC. Dibujar su curva de transferencia de forma
  analítica y mediante simulación.}
\mkexercise{La figura muestra la topología básica de un regulador de
  tensión basado en diodo zener. Estudiar analíticamente y mediante
  simulación el comportamiento de la tensión de salida $V_{out}$
  frente a los cambios en la tensión de entrada $V_{in}$.
