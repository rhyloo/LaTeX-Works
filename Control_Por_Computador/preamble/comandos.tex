%%%%%%%%%%%%%%%%%%%%%%%%%%%%%%%%%%%%%%%%%%%%%%%%%%%%%%%%%%%%%%%%%%%%%%%%
%%																	  %%
%%		         	           Comandos      					      %%
%%																	  %%
%%%%%%%%%%%%%%%%%%%%%%%%%%%%%%%%%%%%%%%%%%%%%%%%%%%%%%%%%%%%%%%%%%%%%%%%

%%Para cambiar la fuente a un fragmento del texto. 
%%usamos \fuente{phv}{loremipsum loremipsum}
\newcommand{\fuente}[2]{\fontfamily{#1}\selectfont #2}	

%%Para saltar una página.
\newcommand\blankpage{		
	\null
	\thispagestyle{empty}%
	\addtocounter{page}{-1}%
	\newpage}
	
\newcommand{\toc}{%
    \tableofcontents
        \clearpage
    \begingroup
        \pagestyle{empty}%
        \cleardoublepage
    \endgroup
%
}

\newcommand{\header}[1]{
    \fancyhead[RO]{\footnotesize\nameref{#1}}
    \fancyhead[LE]{\footnotesize\nameref{#1}}
}

%%%%%%%%%%%%%%%%%%%%%%%%%%%%%%%%%%%%%%%%%%%%%%%%%%%%%%%%%%%%%%%%%%%%%%%%
%%																	  %%
%%		   	TEOREMAS, DEFINICIONES, PROPOSICIONES, ETC.			      %%
%%																	  %%
%%%%%%%%%%%%%%%%%%%%%%%%%%%%%%%%%%%%%%%%%%%%%%%%%%%%%%%%%%%%%%%%%%%%%%%%

%%Nuevo contador {<nombre variable>}{<se reinicia en...>}
\newcounter{example}[section]

%%Nuevo entorno{<nombre entorno>}[<numero de argumentos>]{<\refstepcounter = en referencia a este contador aumenta uno>}{lo que hará con el texto}
%%\newenvironment{ENTORNO}[ARGUMENTOS]{ANTES}{DESPUÉS}
\newenvironment{example}[1]{\refstepcounter{example}%
    {\large{\noindent\textcolor{NavyBlue}{\textbf{Ejemplo~\thesection.\theexample }}}} \vspace{-1em}\par #1}%
    {\par}

\newcommand{\mkexample}[1]{%
 {%
    \begin{example}%
        #1%
     \end{example}%
 }%
}

%\newcounter{solution}[chapter]

%%Nuevo entorno{<nombre entorno>}[<numero de argumentos>]{<\refstepcounter = en referencia a este contador aumenta uno>}{lo que hará con el texto}
%%\newenvironment{ENTORNO}[ARGUMENTOS]{ANTES}{DESPUÉS}
\newenvironment{solution}[1]{\refstepcounter{solution}%
    {\large{\noindent\textcolor{violet}{\textbf{Solución}}}} \vspace{-1em}\par #1}%
    {\par}

\newcommand{\mksolution}[1]{%
 {%
    \begin{solution}%
        #1%
     \end{solution}%
 }%
}

% \newcounter{exercise}[chapter]
% \newcommand{\mkexercise}[1]{%
%  {%
%     \refstepcounter{exercise}
%     \begin{tcolorbox}[sharp corners, colframe=violet]
%         #1%
%     \end{tcolorbox}%
%     \vspace*{0.5em}
%  }%
%  }
\definecolor{bluebox}{HTML}{002E5D}
\newcounter{code}[subsection]
\usepackage{verbatim}
\newcommand{\mkcode}[1]{%
 {%
    \refstepcounter{code}
    \begin{tcolorbox}[sharp corners, colframe=bluebox]
        #1%
    \end{tcolorbox}%
    \vspace*{0.5em}
 }%
}
\newcommand{\mkanscode}[1]{%
 {%
    \refstepcounter{code}
    \begin{tcolorbox}[sharp corners, colback = white]
      \color{gray}
      #1
    \end{tcolorbox}%
    \vspace*{0.5em}
 }%
}


%% Me permite poner en un cuadro teoremas, ejemplos, etc...
\newcounter{definition}[section]
\newcommand{\mkdefinition}[2][\textbf{Definición~\thesection.\thedefinition }]{%
 {%
    \refstepcounter{definition}
    \begin{tcolorbox}[sharp corners, colframe=violet,title ={ \large{\noindent\textcolor{white}{#1}}}]
        #2%
    \end{tcolorbox}%
    \vspace*{0.5em}
 }%
}

\newcounter{demostration}[section]
\newcommand{\mkdemostration}[2][\textbf{Demostración~\thesection.\thedemostration }]{%
 {%
    \refstepcounter{demostration}
    \begin{tcolorbox}[sharp corners, colframe=violet,title ={ \large{\noindent\textcolor{white}{#1}}}]
    \begin{equation*}
        \begin{split}
        #2%    
        \end{split}
    \end{equation*}
    \end{tcolorbox}%
 }%
}


%% Comando que tengo que modificar para diferencialos del resto.
\newcommand{\definicion}[1]{\noindent\textcolor{NavyBlue}{\textbf{Definición #1 }}}
\newcommand{\teorema}[1]{\noindent\textcolor{RedOrange}{\textbf{Teorema #1 }}}
\newcommand{\proposicion}[1]{\noindent\textcolor{PineGreen}{\textbf{Proposición #1 }}}
\newcommand{\lema}[1]{\noindent\textcolor{Tan}{\textbf{Lema #1 }}}
\newcommand{\corolario}[1]{\noindent\textcolor{Salmon}{\textbf{Corolario #1 }}}
\newcommand{\observacion}[1]{\noindent\textcolor{CornflowerBlue}{\textbf{Observación #1 }}}
\newcommand{\ejemplo}[1]{\noindent\textcolor{Lavender}{\textbf{Ejemplo #1 }}}
\newcommand{\demostracion}[1]{\noindent\textcolor{Red}{$\blacktriangleright$ \textbf{Demostracion #1 }}}
%\newcommand{\ejemplo}[1]{\textcolor{RedViolet}{\textbf{Ejemplo #1}}}

%\newcommand{\axioma}[1]{\textcolor{naranja}{\textbf{Axioma #1}}}
%\newcommand{\ejercicio}[1]{\textcolor{amarillo}{\textbf{Ejercicio #1}}}





