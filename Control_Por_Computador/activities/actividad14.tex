\newpage
\subsection{Actividad 14}
Diseñar un controlador analógico tipo PID (PD, PI, PID) para el
sistema servomotor de velocidad por el \textsc{método de Ziegler-Nichols} en
abierto y cerrado (si son posibles) y realizar su discretizacion con
$T=0.05$ aplicando el método trapezoidal y graficar la respuesta escalon
de posición unitario del sistema de control en bucle cerrado del
servomotor de velocidad a través de la herramienta \textsc{rltool}.

Para implementar el método Ziegler-Nichols en bucle abierto,
necesitamos que la respuesta ante entrada escalón sea la respuesta
aproximada de un sistema de primer orden con retardo y que cumpla
$0.15<d/tau<0.6$. Tras calcular los parámetros de nuestro sistema
obtenemos $d/tau = 3 > 0.6$ por lo que no es válido aplicar este
método. En caso de implementar el método de Ziegler-Nichols en bucle
cerrado, no se puede implementar debido a que nuestra planta es
estable independientemente del valor de la ganancia, imposibilitando
el calculo de los parámetros en el límite de la estabilidad, ya que
nunca se alcanza.

\begin{tcolorbox}[sharp corners, colframe=bluebox, title= Método de Ziegler-Nichols,breakable=unlimited]
$>>>$ [Kc,Pm,Wcg] = margin(Gvelocidad)\\
$>>>$ velocidadz = c2d(Gvelocidad,0.05,'zoh')\\
$>>>$ step(Gvelocidadz)\\
$>>>$ T1 = 0.2;\\
$>>>$ T2 = 0.1;\\
$>>>$ d = (3*(T1-T2))/2;\\
$>>>$ tau = T1-d;\\
$>>>$ d/tau\\

  \begin{tcolorbox}[sharp corners, colback = white]
    \color{gray}
\begin{verbatim}
Kc =
   Inf

Pm =
   Inf

Wcg =
   Inf

ans =
3.0000
\end{verbatim}
  \end{tcolorbox}%

 \end{tcolorbox}%


